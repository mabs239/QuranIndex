% !TeX program = xelatex
\documentclass[11pt, a4paper,margin=.7in]{book}
\usepackage[paperwidth=210mm, paperheight=297mm, textwidth=8in,textheight=9in]{geometry}
 
%\usepackage[margin=0in]{geometry}
\usepackage{fancyhdr}
\pagestyle{fancy}
\fancyhf{}
\rhead{الفاظ القرآن}
\lhead{www.bakarabu.wordpress.com}
%\lfoot{ \LaTeX ٹائپ سیٹ  }
\rfoot{\thepage}
\title{الفاظ القرآن}
\author{ابوبکرصدیق}
\usepackage{booktabs}
\usepackage{ltxtable}
\usepackage{polyglossia}
\usepackage[T1]{fontenc}
\usepackage[table]{xcolor}    % loads also »colortbl«
\usepackage{tabularx}
\usepackage{longtable}
\usepackage{bidi}
\usepackage{bidi-longtable}

\setmainlanguage{urdu}
\setotherlanguage{english}
 
\newfontfamily\urdufont[Script=Arabic]{_PDMS_Saleem_QuranFont}
\newfontfamily{\englishfont}[Mapping=tex-text, Scale=1, Script=Latin, Color=gray]{_PDMS_Saleem_QuranFont}
\newfontfamily{\entryfont}[Mapping=tex-text, Scale=1, Script=Latin, Color=black]{_PDMS_Saleem_QuranFont}
 \date{}
\begin{document}
\maketitle
 
\newcolumntype{A}{>{\hsize=.5in\raggedleft \urdufont\setRL}X}
\newcolumntype{B}{>{\hsize=3in\raggedright\urdufont\setRL}X}
%\newcolumntype{C}{>{\hsize=.1in\raggedright\englishfont\setLR}X}
%\newcolumntype{C}{>{\hsize=.1in\raggedright\entryfont\setLR}X}
%\newcolumntype{D}{>{\hsize=.1in\raggedright\englishfont}X}
%\newcommand{\Lt}[1]{\beginL #1\endL} % Left
\LTXtable{\textwidth}{dict.tex}

\end{document}